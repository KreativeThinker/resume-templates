\documentclass[10pt, a4paper]{styles}
\geometry{left=1.2cm,right=1.2cm,marginparwidth=6.8cm,marginparsep=1.2cm,top=1.cm,bottom=1.2cm}
\usepackage[default]{lato}
\begin{document}

\name{Name} 
\tagline{Tag 1 $\bullet$ Tag 2 $\bullet$ Tag 3}%
\personalinfo{%
    \mailaddress{\href{mailto:email@email.com}{email@email.com}} 
    \linkedin{\href{https://linkedin.com/in/username}{linkedin.com/in/username}}\\\smallskip
    \phone{+91 XXXXX XXXXX} 
    \github{\href{https://github.com/username}{github.com/username}} 
    \website{\href{https://website.com}{website.com}}
    \location{location}% 
}
\makecvheader

% -------------------------------------------------------------------------------------------------- %

% Professional Experience - Reverse chronological and company/project specific information of what you did.
\cvsection{Experience}
    \company
        {Position}
        {Company}
        {Location}
        {From -- To}
        \begin{itemize}
          \item What you developed and achieved in this role.
          \item Use action verbs and quantify your achievements where possible.
          \item Follow the STAR method (Situation, Task, Action, Result) to structure your bullet points.
        \end{itemize}

    \company
        {Position}
        {Company}
        {Location}
        {From -- To}
        \begin{itemize}
          \item What you developed and achieved in this role.
          \item Use action verbs and quantify your achievements where possible.
          \item Follow the STAR method (Situation, Task, Action, Result) to structure your bullet points.
        \end{itemize}


% -------------------------------------------------------------------------------------------------- %

\cvsection{Projects}
    \project
        {Title}
        {https://github.com/username/repo}{username/project}
        {Tech Stack Used, e.g., Python, FastAPI, React, Docker, etc.}
        \begin{itemize}
            \item Developed a full-stack application that does something interesting.
            \item Implemented features such as X, Y, and Z, resulting in A, B, and C.
            \item Used technologies like X, Y, and Z to achieve the project goals.
        \end{itemize}
    \dividergray
    
    \project
        {Title}
        {https://github.com/username/project}{username/project}
        {Python, MediaPipe, OpenCV, SVM, NumPy, System Automation, HCI, Parallel processing}
        \begin{itemize}
            \item Developed a full-stack application that does something interesting.
            \item Implemented features such as X, Y, and Z, resulting in A, B, and C.
            \item Used technologies like X, Y, and Z to achieve the project goals.
        \end{itemize}
    

% -------------------------------------------------------------------------------------------------- %
\cvsection{Skills}
\createskills {
    \createskill{Languages}{Python, TypeScript, JavaScript, C, C++},
    \createskill{Backend}{FastAPI, Django, DRF, PostgreSQL, Supabase, Tortoise ORM, Prisma ORM, SQL},
    \createskill{Frontend}{NextJS, React, React Native, Sveltekit, AstroJS, TailwindCSS, ShadCN, HTML, CSS},
    \createskill{DevOps/Tools}{Docker, Docker Compose, Portainer, Railway, GitHub Actions, Typer, Bash, Git, GitHub, CI/CD, Linux},
    \createskill{AI/ML/CV}{MediaPipe, OpenCV, SVM, LangChain, ChromaDB, RAG pipelines, Prompt Engineering},
    \createskill{Concepts}{Microservice Architecture, Operating Systems, REST/RESTful APIs, OOPS, Parallel Processing}
}
\smallskip
\vspace{-20pt}
% -------------------------------------------------------------------------------------------------- %

\cvsection{Awards}
    % Follow Reverse chronological order, with the most recent achievements first.
    \cvachievement{\faCode}{ {\bfseries Position -- Event} \hfill { \color{accent} \faCalendar}  \hspace{0.5em}{Month Year}}
    {About what you did at the event. What you built/achieved. Add link like below if it was a hackathon\par
    \href{https://github.com/username/project}{\githubsymbol \hspace{0.5em}username/project}}
    \par
    
      
    \dividergray
    
    \cvachievement{\faTrophy}{ {\bfseries Position -- Event} \hfill { \color{accent} \faCalendar}  \hspace{0.5em}{Month Year}}
    {Or just add your description here if there is no project or add a link to some proof like a certificate}\par

% -------------------------------------------------------------------------------------------------- %

\cvsection{Education}
    \education
        {Bachelor of Technology, Computer Science}
        {Vellore Institue of Technology}
        {Chennai, India}
        {2023 -- 2027}

\end{document}
